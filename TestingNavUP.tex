\documentclass{article}

\usepackage[margin=2.5cm,left=2cm,includefoot]{geometry}
\usepackage{hyperref}
\usepackage{array}
\usepackage{enumitem}
\usepackage{graphicx}
\usepackage[section]{placeins}
\usepackage{titlesec}

% Set depth for sections (one more than usual)
\setcounter{secnumdepth}{4}

% Set paragraphs to be the 4th section depth
\titleformat{\paragraph}
{\normalfont\normalsize\bfseries}{\theparagraph}{1em}{}
\titlespacing*{\paragraph}
{0pt}{3.25ex plus 1ex minus .2ex}{1.5ex plus .2ex}

% Header and footer
\usepackage{fancyhdr}
\pagestyle{fancy}

\rhead{COS301 - \LaTeX}
\lhead{Team Broadsword Access}
\fancyfoot{}
\fancyfoot[R]{Page \thepage}

\renewcommand{\headrulewidth}{2pt}
\renewcommand{\footrulewidth}{1pt}
%

\begin{document}

	\begin{titlepage}
		\begin{center}

			\line(1,0){400}\\
			[6mm]
			\huge{
				\bfseries Testing NavUP System
			}\\
			[2mm]
			\line(1,0){300}\\
			[15mm]
			\textsc{\large NavUP}\\
			[7.5mm]
			\textsc{\large University of Pretoria - Team Broadsword - Access}\\
			[20mm]
			\large{\textbf{Created By:}}\\
			[2mm]
			\large{	
				\href{https://github.com/KeatonPennels}{Keaton Pennels - 14373018}\\
				\href{https://github.com/JoshC117}{Joshua Cilliers - 14267196}\\
					% add your names here				
			
				}
	

		\href{https://github.com/KeatonPennels/COS-301-Broadsword-Access}{\textsc{\Large GitHub Repository - Team Broadsword-Access}\\[2mm]
		  For more information, please click here}

		\end{center}
		\begin{flushright}
			\textsc{\large 22 April 2017}
		\end{flushright}
	\end{titlepage}

	\cleardoublepage
	\thispagestyle{empty}
	\tableofcontents
	\cleardoublepage

	\thispagestyle{empty}
	\listoffigures
	\cleardoublepage
	\setcounter{page}{1}
	\section{Introduction}\label{sec:intro}

				This chapter of the document aims to present the findings of the indepth testing done on the Gladios instance of the NavUP application. A test model was used for the various specifications of the core functions and innovations implemented.

		\subsection{Definitions, Acronyms, and Abbreviations}\label{subsec:daa}
			\begin{table}[h!]
				\centering
				\caption{Table of Definitions, Acronyms, and Abbreviations used in this document}
				\label{tab: Table 1}
				\begin{tabular}{| m{4cm} | m{12cm} |}
					\hline
					\textbf{Term} & \textbf{Definition} \\
					\hline
					\hline

		
					\hline
					

				\end{tabular}
			\end{table}


		\subsection{Overview}\label{subsec:overview}
			The remainder of this document will consist the test model used during testing, to functional requirements tested as well as the non-functional requirements tested.\\


	\cleardoublepage

	\section{Test Model}\label{sec:overall-Test Model}
	\section{Functional Requirements}\label{sec:overall-functional}
		This chapter aims to give an overview of the entire NavUP system. The system will be contextualised in order to demonstrate the basic functionality of the system as well as demonstrate how the system interacts with other systems. It will also describe the levels, or types, of users that will utilise the system and describe the functionality that is available to said user. At the end of this chapter, the constraints and assumptions for the system will be addressed.

		\subsection{Service Contracts}\label{subsec:overall-contracts}
			NavUP is currently envisioned as a native mobile application that serves to navigate users around the Hatfield campus of the University of Pretoria. This application will be utilised by students, staff and visitors to the Hatfield campus in order to find their way around.\\
			
		\section{Non - Functional Requirements}\label{sec:overall-nonfunctional}
		
		\subsection{Security}\label{subsec:overall-security}
		    \textbf{\underline{Password Protection}}\newline
		    \textbf{Action:} Attempt to circumvent the password protection\newline
            \textbf{Expected Result:} The password protection should be secure and impervious to circumvention.\newline
            \textbf{Actual Result:} The password protection at the login screen is secure and doesn't allow users to login unless their password is correct. The Profile Management page however has a loophole when changing passwords though, in that it allows users to change passwords even if the password they enter into the Old Password field (for authentication) is incorrect.\newline
            \textbf{Possible Improvements:} The app should lock down the profile if the password is entered incorrectly too many times. The profile can then remain locked until an authentication email is sent to the user allowing them to unlock their profile. \newline
            \textbf{Mark:} 7/10\newline
            
            \textbf{\underline{SQL Injection}}\newline
		    \textbf{Action:} Attempt to inject SQL code into the database\newline
            \textbf{Expected Result:} The SQL injection should be blocked by escaping the special characters entered.\newline
            \textbf{Actual Result:} The SQL injection couldn't be tested because none of the data entered is persisted to a database. However, no code was found to sanitize harmful characters from user input before passing it to the user module. \newline
            \textbf{Possible Improvements:} The app could escape characters before passing it to the user module. \newline
            \textbf{Mark:} 2/10\newline

        \subsection{Stability}\label{subsec:overall-stability}
		    \textbf{\underline{Crash Testing}}\newline
		    \textbf{Action:} Attempt to crash the Access module through overloading input.\newline
            \textbf{Expected Result:} The app will remain running.\newline
            \textbf{Actual Result:} The app avoided crashing but was slowed down to an extent by the large amount of input.\newline
            \textbf{Possible Improvements:} The app should implement a character limit on input fields to fully guard against this issue.\newline
            \textbf{Mark:} 8/10\newline
            
        \subsection{Accessibility}\label{subsec:overall-accessibility}
		    \textbf{\underline{Support for disabilities with routes}}\newline
		    \textbf{Action:} Does the app provide an option for routes that are friendly to people with disabilities?\newline
            \textbf{Expected Result:} The app does provides such a feature with routes that avoid stairs and follow wheelchair-friendly routes.\newline
            \textbf{Actual Result:} The app does not provide such a feature.\newline
            \textbf{Mark:} N/A\newline
            
        \textbf{\underline{Support for colour-blindness}}\newline
		    \textbf{Action:} Does the app provide an option to present the User Interface in a manner that is colour-blind friendly?\newline
            \textbf{Expected Result:} The app has options for different kinds of colour-blindness.\newline
            \textbf{Actual Result:} The app doesn't provide support for different kinds of colour-blindness.\newline
            \textbf{Mark:} N/A\newline
            
	    
	    	\subsection{Performance Testing}\label{subsec:performance}
		A IPhone 6 with the below specifications were used in order to conduct the various performance testing \newline
		
		Specifications: Operating System – iOS 10.3 \newline
		  Chipset – Apple A8 \newline 
		 CPU – Dual Core 1.4 GHz \newline
		 RAM – 1GB DDR3  \newline
		 Storage – 32GB \newline
		 
		 The below table displays the performance statics gathered when running the NavUP application on the above device. The respective functionality was conducted 10 times and then an average of these cases were recorded in the Measure column. Recorded statistics were taken from the XCode debugging suite.
		 
		 \begin{table}[h!]
		 	\centering
		 	\caption{Table of collected data for Performance testing}
		 	\label{tab: Table 2}
		 	\begin{tabular}{111}
		 		\hline
		 		\textbf{Functionality} & \textbf{Description} & \textbf{Measurement}\\
		 		\hline
		 		\hline
		 		Application launch &	Time taken for home screen to load when application is opened & 1.42 seconds \\ 
		 		 & Initial memory usage when application is launched  & 118.3 MB  \\ 
		 	    &	Initial percentage of CPU used when application is launched &	0.8 \% \\
		 		 & Amount of data read when application is launched  &	40 KB \\
		 		 & Amount of data read when application is launched  &	300 KB \\
		 		 Registration  & 	Time taken to register a new user & 	0.7 seconds \\
		 		 &  Memory increase when new user is registered	 & 57.9 MB \\
		 		  & Amount of data written when new user is registered & 	1.2 MB \\
		 		 Profile Management & 	Time taken to update user details & 	0.2 seconds \\
		 		  & Memory increase when new user is registered	 & 10.6 MB \\
		 		  & Amount of data read when user profile is displayed & 	200 KB \\
		 		 &  Amount of data that was written when user details were updated & 	0.64 MB \\ 
		 		 
		 		
		 		
		 		
		 		\hline
		 		
		 		
		 	\end{tabular}
		 \end{table}
		 
		 	\cleardoublepage
		
		We were unable to test the navigation with relation to performance as its functionality was not available to us. It was estimated that this would incorporated a large part of the applications performance requirements in terms of memory and CPU usage \newline
		
		\textbf{Mark =} 9/10 \newline
		Comment: The Gladios instance of the NavUP application performs well in terms of time taken for the application the execute user tasks. As this is the majority concern for all 3 user groups, this instance meets the requirements in terms of performance. The memory and CPU time required by the application is in general quite low which allows it to perform well even when other applications are running on a device. It would have been preferable if the performance requirements of navigation could have been tested in order to gauge how the application performs under general usage conditions.
		
		\subsection{Usability Testing}\label{subsec:usabilty}  
		
		The Gladios instance of NavUp usability testing was evaluated in terms of 3 key components: \textbf{Effectiveness} (Whether or not the user’s goal was achieved), \textbf{Efficiency} (Whether a user’s goal is achieved in a proficient manner) and \textbf{Aesthetics} (Whether the look and feel adhered and/or enhances applications purpose). \newline
		
		The user’s goals that were taken into consideration was that of all 3 different user groups of the NavUP application. All tasks were derived from the model Requirements Specifications. Upon completion of these 3 user groups interacting with the application, the following results were gathered. \newline
		
		\textbf{Effectiveness} – Guest and General users were able to carry out all tasks related to the core functionality required of the NavUP application whereas Admin users were not provided with required User and GIS administrative functionality. This was not seen as a major shortcoming in terms of effectiveness as the majority of administrative functionality would take place on the Web instance of the NavUP application. Tasks involving Add On functionality were sparsely implemented for all 3 user groups with only basic navigation functionality being provided to the users \newline
		
		\textbf{Efficiency} – The application provides a simple user interface and requires only basic touch, swipe and type interactions. The majority of functionality is reachable to the user within the range of 3 to 5 clicks/touches. Where the app lacks in efficiency is in two areas:  1) a call to action for the user upon launching the application was absent. The user was required to open a descriptionless menu tab in order being interaction and 2) providing guiding prompts to direct the user’s interaction with the application. Users were not informed of incorrect inputs and struggled when registering, logging in and searching.\newline
		
		\textbf{Aesthetics} – The user is presented with interface whose colour scheme is in conjunction with that of the University of Pretoria. This along with the structure of components within the application being similar to that of other popular navigation applications such as Google Maps and Waze Maps ensured that interactions would be intrinsically familiar to the user. All menus and search interactions were formatted consistently throughout the applications which prevented confusion and allowed for the user to become familiar with the user interface \newline
		
	\textbf{Mark =} 7/10 \newline
	
		Comment:  All three user groups have access to almost all of the core functionality and some of the add on functionality which facilitates the completion of user goals and tasks. Minor adjustments such as improvements to the navigation functionality to allow for a more in-depth searching of locations and route calculations. Error messages and user prompts can also be more elegantly designed in order to facilitate correct interactions with the application. Other than these few points, the Gladios instance of the NavUP application is generally usable from a user perspective and provides an enjoyable interactions 



\end{document}

